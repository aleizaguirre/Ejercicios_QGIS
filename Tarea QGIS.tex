

\documentclass[11pt,spanish]{article}
\usepackage[T1]{fontenc}
\usepackage{selinput}
\SelectInputMappings{%
  aacute={á},
  ntilde={ñ},
}
%%\usepackage[latin9]{inputenc}

\usepackage{array}
\usepackage{float}
\usepackage{amsmath}
\usepackage{graphicx} 

\title{Ejercicios QGIS} 

\date{\vspace{-5ex}}

\begin{document}
\maketitle


\section{Mapas}
El siguiente mapa muestra la población de las comunidades de Chicago por quintiles según datos del Censo 2010.
\begin{table}[H]
 \centering
 \caption{Población}
\begin{tabular}{c}
\includegraphics[scale=0.35]{Chicago_Poblacion1.jpg}
\end{tabular}
\end{table}

El siguiente mapa muestra el total de crímenes en la diferentes comunidades de Chicago. El color refiere al quintil de la distribución (mas oscuro mayor valor) y la altura al valor absoluto de la variable. \\
\begin{table}[H]
 \centering
 \caption{Mapa Crimen}
\begin{tabular}{c}
\includegraphics[scale=0.35]{Crimen.png}
\end{tabular}
\end{table}

El siguiente mapa muestra los precios (per capita) de alquiler de Airbnb en la diferentes comunidades de Chicago. El color refiere al quintil de la distribución (mas oscuro mayor valor) y la altura al valor absoluto de la variable. 

\begin{table}[H]
 \centering
 \caption{Mapa  Precios}
\begin{tabular}{c} 
\includegraphics[scale=0.35]{Precios.png} 
\end{tabular}
\end{table}
\section{Gráficos}

El siguiente gráfico es un Histograma de Frecuencias Absolutas y  muestra la cantidad de comunidades (observaciones) según rango de precios por persona para el año 2015.
\begin{table}[H]
 \centering
 \caption{Precios}
\begin{tabular}{c}
\includegraphics[scale=0.7]{Hist_precios.png}
\end{tabular}
\end{table}


El siguiente gráfico es un gráfico de dispersión y muestra la asociación positiva que existe entre el nivel de pobreza y de desempleo.
\begin{table}[H]
 \centering
 \caption{Pobreza vs. Desempleo}
\begin{tabular}{c}
\includegraphics[scale=0.7]{Plot_pobreza_desocu.png}
\end{tabular}
\end{table}

El siguiente gráfico es Histograma de Frecuencias Absolutas en 2 dimensiones que muestra el total de observaciones (comunidades) según rango de ingreso per cápita y número de crímenes.
\begin{table}[H]
 \centering
 \caption{Ingreso Vs. Crímenes}
\begin{tabular}{c}
\includegraphics[scale=0.7]{2D_ingreso_crimenes}
\end{tabular}
\end{table}

\end{document}